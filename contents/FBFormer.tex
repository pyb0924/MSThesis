% !TEX root = ../main.tex

\chapter{基于X线影像和先验的超稀疏CT重建}

\section{引言}
自CT被发明以来,降低剂量一直是CT领域的一个重要研究方向。

在此类使用直接深度学习方法进行重建的研究中,有一类比较特殊的问题:超稀疏(Ultra-sparse View)CT重建,或者说使用若干X线定位片生成CT影像的问题。减少投影角度是LDCT的一种实现方式,通常只包含100个投影角度以下的投影角度。而正常剂量的CT,使用FBP算法重建需要几百甚至上千个角度。对于这类稀疏角度CT重建问题,使用MBIR或者深度学习算法更为适合。而其中的极端情况,即仅包含几个投影角度的在深度学习算法出现之前,由于投影图包含的信息过少,几乎没有办法得到有效的结果。得益于神经网络强大的拟合能力,目前也已经有了不少这类问题的研究成果。Shen等人首先提出了PatRecon模型,使用单病人的样本通过二维X线影像生成三维CT断层数据,并且对不同数量扫描角度的X线影像样本做了对比实验,在不同部位的样本上证明了深度学习方法对这一问题的可行性\cite{shenPatientspecificReconstructionVolumetric2019}。Ying等人提出了X2CT模型,该方法利用两个正交的X线影像样本生成CT影像,结合生成对抗损失和投影重建损失函数,取得了比较成熟的效果\cite{yingX2CTGANReconstructingCT2019}。由于在该问题中,输入图像为若干张二维图像,而输出则为三维影像,因此在网络结构的设计上必然存在从二维向三维转换的过程,Ying等人也提出了全新的特征转换方式。Jiang等人使用条件变分自编码器(Conditional Variational Autoencoder,CVAE-GAN),从单个X线影像重建了三维CT影像\cite{jiangReconstruction3DCT2021}。Shen等人提出了几何信息图像重建(Geometry-Informed Image Reconstruction,GIIR),通过将不同投影角度的X线影像特征解耦的方式,实现了一个两步骤的三维CT重建\cite{shenGeometryinformedDeepLearning2022}。

\section{基于Transformer和特征反投影的超稀疏CT重建}
\subsection{FBFormer网络结构}

\subsection{混合的PVT和UNet生成器}

\subsection{特征反投影模块}

\subsection{损失函数}

\section{实验数据与实施细节}

\section{实验结果与分析}

\section{讨论}

\section{本章小结}

